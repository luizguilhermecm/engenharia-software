 \section{Introdução}

  \subsection{Objetivo}
  Este documento tem por objetivo a especificação dos requisitos levantados junto a JCI de Londrina. Servindo também como instrumento para os desenvolvedores e membros 
  da JCI para elucidar questões referentes aos requisitos levantados.

  \subsection{Escopo}
    
   O projeto do sistema da JCI de Londrina visa o desenvolvimento de um sistema para cadastro de voluntários e entidades, entidades essas que atuam em várias áreas como creches, 
   cuidados a pessoas da melhor idade, cuidado de deficiêntes, e os voluntário que querem ajudar tais entidades.\\
   O sistema tendo esses dados irá cruzar as informações entre voluntários e entidades e mostrar-la em uma área do sistema que será de acesso exclusido de membros da JCI
   da JCI, após o cadastro de cada novo voluntário ou entidade será mostrado ao membro da JCI os cruzamentos entre aquele novo cadastro e os dados já existentes no 
   banco de dados do sistema.

   O sistema permitirá aos membros da JCI criarem um histórico de voluntariado para cada voluntário ou entidade.\\
   Os dados cadastrados no sistema por voluntários ou entidades estarão disponíveis apenas para membros da JCI, nenhum voluntário ou entidade poderá ter acesso a nenhuma 
   informação que não seja o seu próprio cadastro.

  \subsection{Visão Geral}
  
  No capítulo 2 será feita uma descrição geral dos usuário do sistema. No capítulo 3 serão descritos os requisitos funcionais que haverá no sistema. No 
  capítulo 4 será descrito os requisitos de interface. No capítulo 5 é descrito os requisitos de qualidade e as restrições do sistema serão descritas no capítulo 6.

  \section{Usuários}
  Os usuários do sistema serão aqueles que irão interagir de alguma forma com o sistema, sejam os voluntários ou entidades. Também está definido como usuário os Membros da JCI que acesso como administrador do sistema.

   \subsection{Voluntário Pessoa Física}
   O voluntário pessoa física se cadastrará voluntáriamente no sistema, sendo identificado pelo seu CPF.
   \subsection{Voluntário Pessoa Jurídica}
   O voluntário pessoa jurídica se cadastrará voluntáriamente no sistema, sendo identificado pelo seu CNPJ.
   \subsection{Entidade}
   A entidade se cadastrará voluntáriamente no sistema, sendo indentificada pelo seu CNPJ.
   \subsection{Membro JCI}
   O membro JCI terá acesso a todos os dados cadastrados, sendo identificado por um login e senha, sendo que, não haverá hierarquia entre tais usuários. 

  \section{Requisitos Funcionais}
    \subsection{Cadastro de Voluntário}
      \subsubsection{Pessoa Física}
      O cadastro de pessoa física no sistema será feita pelo voluntário e que deverá obrigatóriamente ter um CPF para identifica-lo, esse CPF será validado pelo sistema, seu cadastro conterá:
        \begin{itemize}
        \item Nome
        \item CPF
        \item Email
        \item Telefone
        \item Como ficou sabendo da JCI/Projeto Canal de Voluntários?
        \item Tem experiência com voluntáriado?
        \item "Público":
          \begin{itemize}
          \item Crianças
          \item Adultos
          \item Melhor idade
          \item Adolescentes
          \item Portadores de necessidades especiais
          \item Outro (campo aberto)
          \end{itemize}
        \item Áreas de atuação:
          \begin{itemize}
            \item Jurídica
            \item Administrativa
            \item Recreação
            \item Saúde
            \item Educação
            \item Manutenção
            \item Doação
            \item Outro (campo aberto)
          \end{itemize}
        \item Disponibilidade de tempo
        \item Outras informações (campo aberto)
        \end{itemize}

      \subsubsection{Pessoa Jurídica}
      O cadastro de pessoa jurídica no sistema será feita pelo voluntário e que deverá obrigatóriamente ter um CNPJ para identifica-lo, um membro JCI irá verificar 
      a validade deste CNPJ junto ao cadastro nacional, seu cadastro conterá:
        \begin{itemize}
          \item Razão Social
          \item CNPJ
          \item Inscrição Estadual (não obrigatório)
          \item Email
          \item Telefone
          \item Endereço
          \item Área de atuação da empresa
          \item Contato (nome do responsável)
          \item Outras informações (campo aberto)
        \end{itemize}

    \subsection{Cadastro de Entidade}
    O cadastro de entidade no sistema será feito pela entidade e deverá obrigatóriamente ter um CNPJ para identifica-la, esse cadastro ficará inativo no sistema até 
    que um membro JCI verifique a veracidade das informações da entidade e aprove ou desaprove o cadastro, seu cadastro conterá:
    \begin{itemize}
        \item Nome
        \item CNPJ
        \item Endereço
        \item Nome do responsável
        \item Email
        \item Telefone
        \item Site
        \item Data de fundação
        \item Público atendido:
          \begin{itemize}
          \item Crianças
          \item Adultos
          \item Melhor idade
          \item Adolescentes
          \item Portadores de necessidades especiais
          \item Outro (campo aberto)
          \end{itemize}
        \item Número de beneficiados
        \item Áreas de atuação: 
          \begin{itemize}
            \item Jurídica
            \item Administrativa
            \item Recreação
            \item Saúde
            \item Educação
            \item Manutenção
            \item Doação
            \item Outro (campo aberto)
          \end{itemize}
        \item Recebe apoio? (sim/não)
          \begin{itemize}
            \item Municipal
            \item Estadual
            \item Federal
            \item Particular
          \end{itemize}
        \item Horários de atendimento:
      \end{itemize}

    \subsection{Modificação do Cadastro pelo Voluntário ou Entidade}
    O sistema permitirá ao voluntário ou entidade cadastrado alterar seus dados ou excluir seu cadastro do sitema da JCI, essa modificação será feita pelo 
    voluntário ou entidade após se identificar com CPF ou CNPJ, assim tendo acesso aos dados do seu cadastro.

    \subsection{Cruzamento de Dados}
    O cruzamento de dados será feito de forma automática. E somente os membros JCI terão acesso a essa funcionalidade.

    Toda vez que um novo voluntário ou entidade é adicionada, no caso de ser uma entidade o sistema fará um cruzamento entre as necessidades da entidade 
    e os voluntários já cadastrados e retornando aqueles que combinarem, no caso voluntário o sistema irá cruzar sua área de atuação e público com as entidades 
    existentes no sistema e retornando aquelas que combinarem.

    \subsubsection{Dados para teste}
    Para teste do sistema a JCI poderá passar cadastros válidos para serem utilizados como caso de teste.

    \subsection{Inclusão de Histórico}
    O sistema permitirá a inclusão de histórico de voluntáriado em cadastros existentes, sendo possível ser adicionado apenas por membros JCI, que poderão entrar 
    no cadastro do voluntário ou entidade e lá adicionar informações em um campo aberto para texto.

    \subsection{Busca Personalisada}
    O sistema permitirá realizar buscas personalizadas entre os cadastros existentes, seja para encontrar um voluntário ou entidade para incluir histórico ou para 
    localizar todos os voluntários com determinados interesses ou entidades com determinadas necessidades. Apenas membros JCI terão acesso a essa funcionalidade. 

    \subsection{Modificação de Cadastros}
    O sistema permitirá que um Membro JCI modifique qualquer informação cadastrada por voluntário ou entidade.

  \section{Requisitos de Interface}
  O sistema será desenvolvido como um sistema WEB, com uma interface clara e objetiva, será priorizado na interface manter a clareza das informações dispostas na 
  interface, fazendo com que o usuário não tenha que perder tempo procurando ou advinhando o que a interface está mostrando. O uso do sistema deverá ser o mais 
  intuitivo possível para facilitar seu uso.
  
  \section{Requisitos de Qualidade}
  A qualidade do sistema será prioridade no que diz respeito funciolidade.

  A velocidade de resposta em buscas ou cruzamento de dados deverá ser de no máximo 5 segundos. O tempo de resposta que não incluam o banco de dados deverá ser de 
  no máximo dois segundos, sempre levando em consideração a velocidade de conexão do usuário.

  O sistema deverá rodar nos principais navegadores que estão no mercado (Google Chrome, Mozilla Firefox, Safari, Internet Explorer), priorizando as versões 
  mais atuais de cada navegador.

  Todos os requisitos citados na sessão de requisitos funcionais são obrigatórios no sistema.

  \section{Restrições}
  O sistema terá restrições no que diz respeito a navegadores de versões descontinuadas, como por exemplo: \textit{Internet Explorer 6}.

  Para implantação do sistema deve-se ter um dominio válido e um servidor para ser alocado, essas necessidades do sistema serão de responsabilidade da JCI Londrina em contrata-los e compra-los se necessário, a \textit{G8k} não se envolverá nesse processo. 
